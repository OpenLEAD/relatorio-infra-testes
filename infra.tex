Não existe, no momento, nenhuma infraestrutura no local para as
realização dos testes, sendo necessário a implementação da infraestrutura civil,
elétrica e de suporte para os testes.

Primeiramente, o objetivo principal dos testes consistia em realizar um processo
de pintura automática, simulando o processo de metalização e com forte apelo
visual para o cliente. Entretanto, como será exposto a seguir, a infraestrutura
necessária para o processo é extensa e os paramêtros realmente testados não são
explícitos.

\subsection{Pintura}

Sistema de Pintura: para a utilização de um sistema de pintura é necessária a utilização de sistemas de: 
compressão de ar, purificação de ar, compressão de tinta, isolamento e exaustão.

Compressão de ar: O ar necessário para o sistema deve ser limpo e seco. Para isso é necessário utilizar 
um compressor de diafragma, no qual não existe contato do pistão com o ar a ser comprido ou um filtro de 
óleo. A umidade também afeta o sistema e é necessário a utilização de filtros desumidificadores e condensadores 
de água. Entretanto para o propósito em questão, acredito que apenas um compressor de diafragma seria suficiente, 
pois não há necessidade de uma pintura de qualidade.

Compressão de tinta: As pistolas de tinta podem ser dividas, basicamente, em dois grandes grupos: pistolas a 
gravidade e pistolas automáticas. A primeira classe possui um recipiente onde é
armazenada a tinta e é acoplada à pistola. Essa característica restringe o
movimento do conjunto, forçando que o efetuador fique orientado sempre com o
recipiente na vertical. Essa restrição adicional é desnecessaŕia para o nosso teste e, por isso, desclassifica esse tipo de pistola como um possível dispositivo para a realização do mesmo. A segunda 
categoria utiliza, além da alimentação de ar comprimido, uma alimentação de tinta proveniente de um 
tanque de pressão de tinta.

Outro fator importante é o isolamento da "cabine de pintura" e a correta exaustão do ar do particulado em 
suspensão. A pistola de tinta pulveriza a tinta em direção à peça a ser pintada e parte dessa tinta fica 
em suspensão no ambiente, impregnando as paredes e objetos próximos. A presença humana nesse ambiente requer 
máscaras de respiração e proteção para olhos e roupas, e qualquer equipamento
eletrônico próximo seria danificado, incluindo computadores e o próprio
controlador e manipulador. Sendo assim, é necessário o isolmanento da
área onde será realizada a pintura e proteção do manipulador. Outro ponto a ser
considerado  é a realização de exaustão, filtragem e
descarte corretos da tinta, para que não se danifique nenhuma estrutura
adjacente ao local de testes (LEAD e prédios vizinhos).

A área útil de metalização realizada pela pistola é de 3mm e as pistolas automáticas são para pinturas de grandes 
superfícies, não sendo possível uma cobertura tão precisa. Portanto, o propósito
do teste é prejudicado e a pintura serviria apenas como um indicativo visual,
não representando a real trajetória do efetuador e nem a real cobertura
realizada.

Portanto, não acredito que o investimento de capital, tempo e mão de obra para a construção de uma infraestrutura 
de pintura para o teste de viabilidade técnica para o projeto EMMA 1 justifique os benefícios alcançados por esse 
tipo de procedimento. Outro fator limitante é o espaço ocupado por todos os equipamentos em comparação com o 
espaço disponível no momento, como será apresentado a seguir. Também será apresentado alternativas de testes 
possíveis para ser realizados com o espaço e tempo disponíveis, que consigam avaliar os requisitos mínimos para 
a viabilidade técnica do processo de metalização realizado pelo sistema proposto.  

\subsection{Configuração mínima}

A partir do espaço disponível, foi realizado um esboço de uma configuração
mínima de testes. Como pode ser observado na figura \ref{fig::planta}, a
acomodação dos itens necessários se dá de maneira apertada. A
configuração mínima necessária foi considerada como: uma pá em escala 1:1, uma
seção do trilho primário servindo apenas como ponto de apoio e pivoteamento,
trilho secundário de 2,75m, manipulador MH12 e seu controlador DX200.

\begin{figure}[h!]
\centering
	\includegraphics[width=0.9\columnwidth]{figs/espaco/Montagem_Base_LEAD}
	\caption{Espaço disponível e possível disposição dos elementos necessários
	para os testes.}
	\label{fig::planta}
\end{figure}


Pode-se observar que o robô não tem seu espaço de trabalho totalmente livre. As
paredes ficam dentro de seu alcance, representando um perigo estrutural
para o ambiente,  existindo apenas uma posição para a sua base que possibilita
um espaço livre no semi circulo frontal do robô, essa seria a unica área que teríamos para realizar testes 
antes de simular o coating da pá. Esse cenário não é o ideal, uma vez que até o teste final, 
é necessário um espaço para livre movimentação sem risco de colisões. 

\subsection{Demanda Elétrica}

O ambiente de testes deve ser dimensionado para suportar uma demanda, além da
consumida pela iluminação e o uso de aproximademente 4 computadores, de 1.5 kVA
relacionada ao controlador DX200 e o manipulador MH12. 
Esses dispositivos operam em sistema trifásico, suportando 240/480/575 V a 50/60. 
O compressor de ar incluíria %TODO potencia do compressor
na demanda total.
Todos os elementos devem ser considerados como de uso
simultâneo. 

A viabilidade de um sistema de resfriação ou ventilação não foi citado pois
depende das características estruturais do projeto. 

\subsection{Base do manipulador MH12}
O manipulador deve possuir uma base padrão instalada no ambiente de testes, para
que seja possível a realização de testes, modificações e reparos na base e
trilho propostos. 
A princípio, não é necessário que a base suporte o robô em
movimento, entrentanto caso haja espaço suficiente, essa possibilidade é
interessante para os testes relacionados apenas ao manipulador sem que haja
influência da base projetada.


\subsection{Movimentação de equipamentos pesados}
O manipulador MH12 deverá ser movimentado constantemente na área de testes e tem
peso de $130kg$, por isso julga-se necessário a implementação de um sistema de
talha e carro trole. O carro trole pode se movimentar em uma viga I pertencente
a própria estrutura de sustentação do ambiente de testes, ou permanente fixada
no mesmo. A estrutura de movimentação deve permitir o transporte entre do
manipulador a partir de sua base até o local de testes de cobertura, para
acoplamento à estrutura mecânica.


\subsection{Bancada de trabalho}
É necessário um espaço para trabalho para pelo menos 1 pessoa.
Primeiramente foi idealizado a utilização da área ocupada pelo controlador DX200
e a construção de uma bancada de apoio. 
Portanto, afim de se poupar espaço, o
controlador ficaria embaixo da bancada e a pessoa trabalhando nos testes usaria
o computador em pé.

 \subsection{Suporte da pá 1:1}
 O suporte da maquete em escala 1:1 da pá deve, idealmente, o giro da pá em
 torno de seu próprio eixo vertical e horizontal. 
 O movimento no eixo vertical possibilita o teste de cobertura em ambos os lados
 da pá, caso contrário é necessário retirar a pá do suporte e realizar o giro.
 É importante ressaltar que dependendo do tamanho do galpão em que o teste seja
 realizado, não será possível a movimentação da pá, sendo necessário a retirada
 da pá de dentro do ambiente para manobra.
 
 \subsection{Fluxo de equipamentos}
 Excluindo-se a instalação da pá discutida na seção \ref{sec::instal_pa}, apenas
o manipulador MH12 e seu controlador representam objetos de dimensões
significativas que deverão passar pela porta de acesso do galpão de testes.
As dimensões do manipulador são e do controlador DX200 são 800(l)   x  650(p) 
x1000(a). 


\subsection{Instalação da Pá}\label{sec::instal_pa}

A fabricação da maquete da pá em escala $1:1$ está prevista para ser iniciada no
laboratório de esculturas da Escola de Belas Artes da UFRJ (EBA)localizado no
prédio da Reitoria, no Fundão.
Esta etapa prevê o recebimento da matéria-prima e o corte das seções que
fomrarão o esqueleto da maquete. 
Foi recomendado que a etapa de montagem e fabricação fosse realizada no nosso
laboratório, evitando assim o transporte que, devido às dimensões do modelo,  só
poderia ser feita por caminhão que, devido a sua fragilidade, traz risco para a
integridade do modelo.
Assim, deve-se prever o espaço para trabalho de fabricação desta na área externa
do LEAD, bem como o armazenamento adequado do produto final, e também, o
armazenamento provisório da maquete, durante a construção do galpão.

\subsection{Fixação da Base}\label{sec::fix_base}

A fixação da base do manipulador pode ser feita de duas maneiras:

\begin{enumerate}
  \item \textbf{Flanges}: Utlizando flanges de fixação com chumbadores (ver
  seção~\ref{sec::chumbadores}), diretamente entre a estrutura da base e o chão.
  Desta forma, limita-se a liberdade de mudança de posicionamento da base, pois 
  cada posição depende de furação do chão e utlização dos chumbadores, criando \textit{spots} de fixação.
  \item \textbf{Placa metálica}: Desta forma, seria instalada uma placa metálica
  de aço com pelo menos $15~mm$ de espessura, fixada com chumbadores. Assim, a
  fixação da base seria por meio das bases magnéticas dimensionadas para
  a solução final. Isto forneceria liberdade de posicionamento para a base do
  robô em qualquer lugar da placa metálica.
\end{enumerate}

 
 \subsection{Chumbadores}\label{sec::chumbadores}
 
 Os chumbadores são elementos de fixação mecânica compostos por parafuso,
 arruela, jaqueta e cone.
 Para fixação da base do robô, seja por qualquer uma
 das soluções propostas na seção~\ref{sec::fix_base} é necessária a utilização 
 de chumbadores devidamente dimensionados, levando-se em consideração os
 esforços  máximos dinâmicos do robô e a geometria da base.
 Assim, uma estimativa inicial prevê a utilização de parafusos com diâmetro de
 rosca de $16~mm$ e profundidade do furo para a jaqueta de, no mínimo $93~mm$.
 
 Logo, deve-se verificar a capacidade do local da instalação para receber os
 chumbadores, sobretudo verificando se os furos não interferem na passagem de
 tubulação de água ou gás.
 
