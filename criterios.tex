\subsection{Cobertura da pá}
Um dos principais requisitos é a capacidade do sistema de posicionar
corretamente o efetuador em toda a superfície da pá utilizando o manipulador
escolhido (Motoman MH12) e o sistema mecânico composto por trilho e base,
garantindo assim uma total cobertura da superfície a ser revestida.
Caso contrário, o processo de metalização não pode ser finalizado
utilizando a solução proposta como único método de
revestimento. Portanto, um dos critérios a ser analisado é o alcance do
manipulador em todos os pontos ou pontos chave previamente selecionados,
obedecendo a restrição de um ângulo de até $30^o$  em
relação à superfície da pá e uma distância entre $0.23m$ e $0.24m$, até a ponta
da pistola de metalização (ou peça representando a mesma).

\subsection{Payload e vibrações}
O processo de metalização utiliza, atualmente, uma pistola com peso aproximado
de $8,5kg$ e necessita de cabeamento até o sistema auxiliar externo.
Para assegurar a qualidade final do revestimento, o processo tem como requisito
uma velocidade constante de $40m/min$ durante o jateamento. 
Portanto, o sistema base e manipulador deverá ser capaz de realizar uma
trajetória pré-estabelecida com um payload de $8,5kg$, na velocidade
especificada, e sofrendo a ação de um empuxo no sentido contrário à ponta do efetuador.

\subsection{Calibração}
A posição relativa entre o manipulador e ao modelo em escala 1:1 da pá deve ser
calculada a partir da nuvem de pontos gerada pelo sensor Faro Focus X330. 
O sistema de calibração também deve ser capaz de identificar e posicionar as pás
do modelo em escala 5:1 em diversas posições do rotor e ângulo de ataque das
pás.

\subsection{Estrutura mecânica}
Deverão ser verificadas a capacidade de montagem da estrutura mecânica, a
robustez do sistema e seu nível de vibrações, quando aplicados as forças
e torques máximos especificados no \textit{datasheet} do manipulador MH12.
A capacidade de posicionamento da base do robô, assim como o pivoteamento entre
o trilho primário e secundário.

\subsection{Shutter}

A capacidade de desvio do fluxo de particulado, antes da queima, deve ser
avaliado, assim como as alterações na pressão no sistema de alimentação.


