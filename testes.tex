Essa seção apresentará propostas de testes afim de contemplar os critérios
apresentados na seção \ref{sec::criterios}

\subsection{Teste da base mecânica}
Devido ao espaço limitado e as configurações
possíveis de fixação, foi analisado que uma possiblidade de testes é avaliar a rigidez estrutural do trilho secundário, 
considerando uma única posição de fixação e sem a necessidade da presença de bases magnéticas. 
O teste contemplaria apenas a rigidez entre o ponto de pivoteamento e o trilho secundário e entre o 
trilho secundário e a base do robô. Assim como a necessidade e possibilidade de uma fixação da base do 
robô diretamente a superfície de apoio.



\subsection{Teste de cobertura da pá}

Visto que uma infraestrura de pintura não apresentou um custo benefício
interessante, outros métodos devem ser propostos.

\begin{itemize}
  \item \textbf{Toque em pontos específicos:} para avaliar a cobertura, seria
  possível um cenário de testes no qual, com o auxílio de um efetuador, o
  manipulador realizaria toques nas partes extremas da pá. A cobertura total
  seria assumida a partir da hipótese que os pontos amostrados são  pontos
  críticos e suficientes.
  \item  \textbf{Pintura com caneta:} afim de garantir o apelo visual para o
  cliente, seria possível realizar um cenário no qual o manipulador realizaria a
  pintura da pá com a ajuda de uma caneta e um efetuador com um sistema de molas
  para garantir uma tolerância em relação à distância da pá. A caneta teria uma
  espessura mais próxima à área útil de metalização da chama, oferecendo uma
  medida mais realista da cobertura da pá se comparada a um sistema de
  pintura.
  \item \textbf{Pintura por \textit{light painting}:} seguindo o mesmo
  raciocínio da pintura com caneta, o robô realizaria uma pintura virtual da pá,
  utilizando um laser e uma câmera para sobrepor toda a trajetória. Esse cenário
  retira a complexidade mecânica e o risco de danificação tanto da pá, quanto da
  caneta, porém adiciona complexidade de implementação e integração do sistema.
\end{itemize}

\subsection{Teste de Calibração}

A partir da infraestrutura proposta, os cenários de testes da calibração seriam
a calibração em escala 1:1 com a pá em posição fixa e o manipulador sendo
movimentado. A calibração pode ser realizada com o sensor sendo posicionado de
maneiras diferentes, com a presença ou não de oclusão. Para a identificação do
cone e diversos posicionamentos da pá, é possível a utlização da maquete 5:1.

\subsection{Payload e vibrações}

Para os testes de payload e vibrações é prosposto a utilização de um lastro
simulando o peso da pistola e um compressor de ar para simuluar o empuxo gerado
pelo processo. Esse compressor pode ser simples e o laboratório já possui um de
dimensões que não atrapalhariam a configuração proposta. É preciso apenas uma
especificação correta das forças geradas pelo processo para o dimensionamento da
vazão de ar necessária e verificação da conformidade do atual compressor.

\section{Shutter}

O estudo do Shutter ainda está em aberto e necessita de mais discussões com a
empresa parceia Rijeza. É possível que os testes desse dispositivos sejam
realizados nas instalações da Rijeza no sul do páis ou na própria usina de
Jirau.

\subsection{Teste da Base Magnética}
 
 Se selecionada a opção $2$ descrita na seção~\ref{sec::fix_base}, como
 requisito de segurança deve-se testar previamente a capacidade da base
 magnética na estrutura metálica instalada.
 O objetivo do teste é garantir que não haverá desacoplamento da base magnética
 em relação à placa durante os testes dinâmicos do robô.
 
 Para o teste será necessária a utilização de talha manual ou elétrica, com
 capacidade superior à capacidade máxima da base magnética, e uma balança
 suspensa, com o mesmo requisito de capacidade, que medirá a carga máxima
 trativa admissível do imã. 
 Para os esforços tangenciais, calcula-se a força máxima de atrito estático da
 interface, sendo este um resultado conservador para especificar a carga máxima
 tangencial admissível da base magnética.
